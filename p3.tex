\subsection{Cambios realizados sobre la segunda entrega}

En el modelo entidad/relación, se elimina el atributo ``Género relacionados'' de la entidad ``Autor'' y se añade una relación entre esta entidad y la entidad ``Género''. Además, se añade una relación de herencia entre ``Usuario'' y ``Usuario con privilegios''. También se eliminan las relaciones ``Añadir'', ``Modificar'' y ``Borrar'', ya que solo serían necesarias si quisiéramos almacenar qué usuario almacena, modifica o borra una canción determinada. Consideramos que esta información no es propia de nuestro sistema y por la complejidad que implica, decidimos eliminarlas.

\subsection{Diseño lógico}

\subsection{Diseño físico}

A continuación se exponen las secuencias SQL necesarias para la creación de todas las tablas del sistema.

\lstinputlisting[language=SQL]{src/ddsi.sql} 

\subsection{Funcionalidad a nivel de base de datos}

A continuación se exponen los disparadores creados por cada uno de los integrantes del grupo.

\subsection{Subsistema de música}

\lstinputlisting[language=SQL]{src/disparador-jmml.sql} 
